Beam - a 1D series of acoustic signal (dB) per depth (mm). A vertical (along depth) slice of an ultrasound image.

Acceptance region - Usually when we're talkinga bout an R01 we're talking about a an acceptance region located at a specific laterial and axial location.  This is often down the middle, but it doesn't need to be down the middle.


Grating lobes - Off-axis beam artifacts were a significant problem in the early linear array designs,. These were the side product of grating lobes which resulted from ultrasound beams that emulated at predictable angles off-axis to the main beam. Grating lobes are unique to array transducers and are caused by the regular, periodic spacing of the small array elements. When the energy of these lobes is reflected by off-axis structures and detected by the transducer, the signal produced is artifactual and produce "ghost images" blurring the main image. To overcome this problems each individual element has been subdivided into a half wavelength wide. This effectively eliminated the grating lobes by increasing the angle to greater than 90 degrees. Eliminating grating lobes also improves the signal-to-noise ratio by increasing the size of the main lobe energy relative to the background energy. This further improves image contrast.

Insonification -

Modulation -

Center frequency - 
