% (This is included by thesis.tex; you do not latex it by itself.)

\begin{abstract}

  Medical ultrasound is a noninvasive, affordable, portable, and real-time diagnostic modality that provides cross-sectional views of human tissues. Ultrasound beamforming is a widely used approach to acquire and process data from ultrasound probes in a focused manner. However, noises from tissue layering cause artifacts such as off-axis scattering and reverberation clutter and degrade the beamformed images. Recently, frequency-domain multi-layer perceptrons (MLPs) prove effective in suppressing off-axis scattering and improving image contrast. This thesis extends the frequency-domain neural network approach to study the effectiveness of convolutional neural networks (CNNs). A variety of convolutional architectures are proposed, and the contribution of the convolution operation is investigated. Preliminary results show that although hybrid convolutional- and full-connected neural networks can achieve a similar performance compared with MLPs, fully-convolutional neural networks do not perform well because they do not learn additional features. Instead, they approximate full connections by having a large effective receptive field.

\end{abstract}
